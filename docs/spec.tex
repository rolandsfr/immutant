\documentclass[12pt,a4paper]{report}

\usepackage{formatting}
\usepackage{tabularx}
\usepackage{cite}  % optional, improves formatting

\title{LATVIJAS UNIVERSITĀTE
EKSAKTO ZINĀTŅU UN TEHNOLOĢIJU FAKULTĀTE}

\author{Rolands Frīdemanis}


\begin{document}

\begin{titlepage}
    \centering
    
    {\Large LATVIJAS UNIVERSITĀTES\\EKSAKTO ZINĀTŅU UN TEHNOLOĢIJU FAKULTĀTES DATORIKAS NODAĻA\\[5cm]}
    
    {\LARGE \textbf{ INTERPRETATORS PROGRAMMĒŠANAS\\[0.3cm] VALODAI IMMUTANT}}\\[0.5cm]

    \normalsize{KVALIFIKĀCIJAS DARBS DATORZINĀTNĒS}\\[5cm]
    
    \begin{flushleft}
        Autors: Rolands Frīdemanis\\
        Studenta apliecības Nr.: rf23009\\
        Darba vadītājs: ...
    \end{flushleft}
    
    \vfill
    RĪGA, 2025
\end{titlepage}

\selectlanguage{latvian} 
\begin{abstract}
    Mūsdienās liela daļa no vispārīga pielietojuma programmēšanas valodām, satur sematiku, kas paredz, ka datu mainība ir ierasta lieta, tomēr šim pastāv būtisks trūkums, kas ir datu neparedzamība.
    Lai spriešanu par programmas stāvokli padarītu paredzamu, šī darba ietvaros tiek izstrādāta programmēšanas valoda, kuras gramatika un sintakse ierosina noteikta datu nemainību.
    Šis darbs satur valodas specifikāciju un interpretatora arhitektūras dokumentāciju lekserim, parsētājam un abstraktā sintakses koka pārstaigāšanas algoritmam.
    Rezultātā, izmantojot valodu C, tiek izveidota augsta līmeņa, intepretējama, dinamiski tipizēta valoda ar atomāriem datu tipiem.

    \begin{flushleft}
    \textbf{Atslēgvārdi:} interpretators, AST, funkcionālā programmēšana, C valoda, datu nemainība
    \end{flushleft}
\end{abstract} 

\selectlanguage{english} 
\begin{abstract}
    Most modern general-purpose programming languages use grammar and syntax that suggests mutable data being an ordinary matter, which in reality complicates reasoning about program state.
    To make such reasoning predictable, this work explores the design and delelopment of a programming language with explicit syntax and semantics of immutable data. 
    This work contains specification for such language and documentation of the architecture for the lexer, parser and AST-walker of the underlying interpreter.
    As a result, a high-level, interpreted, dynamically typed programming language with only atomic data types is created. The language is implemented in C.

    \begin{flushleft}
    \textbf{Keywords:} interpreter, AST, functional programming, C language, data immutability
    \end{flushleft}
\end{abstract} 

\selectlanguage{latvian} 
\tableofcontents

\newpage
\chapter*{Apzīmējumu saraksts}
\addcontentsline{toc}{section}{Apzīmējumu saraksts} 


\begin{tabular}{ll} 
\textbf{AST} & Abstraktās sintakses koks \\
\textbf{immutability} & Pilnīga datu nemainība (koncepts) \\
\textbf{immutable} & Nemainīgs (datu īpašība) \\
\textbf{read-only} & Tikai lasāmi (dati)
\end{tabular}


\newpage
\chapter*{Ievads}
\addcontentsline{toc}{section}{Ievads}

Datu mainība un to nepārtraukta plūsma no viena mainīga uz otru ir neapstrīdama daļa no lielas daļas programmatūras. Pavisam noteikti var apgalvot, ka datorsistēmu arhitektūru atmiņas koncepcija ļauj rakstīt atmiņā ne vienu vien reizi. Atmiņu var relocēt, izdzēst, pieprasīt lielākus atmiņas gabalus u.t.t. No šī izriet datu mainības esence un līdz šai dienai tā pamata sastāvdaļa iekalta lielā daļā, ja ne visu, programmēšanas un skriptēšanas valodās.   

No otras puses, mainīgiem datiem pastāv īpašība būt neparedzamiem. Kamēr mazas sistēmas spēj tikt galā ar nelielu daudzumu mainīgo, tad lielajās sistēma tas var kļūt par acīmredzamu problēmu. Atmiņa datoram ir viena, tomēr atsaukties uz to var no dažādām vietām dažādos laika brīžos, kas padara spriešanu par datu stāvokli daudz sarežģītāk.  


Programmētāji izmanto iespēju mainīt datus brīvā veidā, jo  programmēšanas valodas un to abstrakcijas ir padarījušas to par tik vienkāršu procesu. Lai gūtu vairāk kontroles pār ipriekš minētas uzvedības, valodas no C saimes, Java, JavaScrip un citas valodas satur gramatikas, kas ierobežo mainīgu datu inicializāciju un to turpmāko vērtību maiņu. Kā piemēru var minēt \texttt{const} atslēgvārdu no valodas C, kas fiksē doto mainīgo un liedz pārrakstīt tā vērtību, tomēr tas tikai daļēji attiecas uz atsauces tipa vērtībām. Kaut arī šāda tipa mainīgais vienmēr atsauksies tikai uz vienu konkrētu atmiņas apgabalu, nav noteikts, ka vērtība, kas tajā atrodas nemainīsies. Līdz ar to, realitātē \texttt{const} atslēgvārds negarantē patiesu datu nemainību, bet tikai noslēdz to uz \texttt{readonly} pieeju.
Neskatoties uz programmēšanas valodu dažādiem centieniem ierobežot datu mainību, tādas problēmas kā neparedzamas mainīgo vērtības pavedienu izpildes laikā, mainīgo aizstājējvārdu nepārdomāta ieviešana un izmantošana, mainīgo nejauša re-inicializācija u.c. joprojām sastāda lielu daļu programmu.\cite{immutability}

Šis darbs izskata jaunas interpretējamas programmēšanas valodas izveidi, kurai pielietota datu nemainības semantika, cenšoties mazināt programmatūras izstrādes problēmas saistītas ar stāvoklu maiņu.

\bibliographystyle{plain} 
\bibliography{references}   

\end{document}

