\documentclass[12pt,a4paper]{report}

\usepackage{formatting}
\usepackage{tabularx}

\title{LATVIJAS UNIVERSITĀTE
EKSAKTO ZINĀTŅU UN TEHNOLOĢIJU FAKULTĀTE}

\author{Rolands Frīdemanis}


\begin{document}

\begin{titlepage}
    \centering
    
    {\Large LATVIJAS UNIVERSITĀTES\\EKSAKTO ZINĀTŅU UN TEHNOLOĢIJU FAKULTĀTES DATORIKAS NODAĻA\\[5cm]}
    
    {\LARGE \textbf{ INTERPRETATORS PROGRAMMĒŠANAS\\[0.3cm] VALODAI IMMUTANT}}\\[0.5cm]

    \normalsize{KVALIFIKĀCIJAS DARBS DATORZINĀTNĒS}\\[5cm]
    
    \begin{flushleft}
        Autors: Rolands Frīdemanis\\
        Studenta apliecības Nr.: rf23009\\
        Darba vadītājs: ...
    \end{flushleft}
    
    \vfill
    RĪGA, 2025
\end{titlepage}

\selectlanguage{latvian} 
\begin{abstract}
    Mūsdienās liela daļa no vispārīga pielietojuma programmēšanas valodām, satur sematiku, kas paredz, ka datu mainība ir ierasta lieta, tomēr šim pastāv būtisks trūkums, kas ir datu neparedzamība.
    Lai spriešanu par programmas stāvokli padarītu paredzamu, šī darba ietvaros tiek izstrādāta programmēšanas valoda, kuras gramatika un sintakse ierosina noteikta datu nemainību.
    Šis darbs satur valodas specifikāciju un interpretatora arhitektūras dokumentāciju lekserim, parsētājam un abstraktā sintakses koka pārstaigāšanas algoritmam.
    Rezultātā, izmantojot valodu C, tiek izveidota augsta līmeņa, intepretējama, dinamiski tipizēta valoda ar atomāriem datu tipiem.

    \begin{flushleft}
    \textbf{Atslēgvārdi:} interpretators, AST, funkcionālā programmēšana, C valoda, datu nemainība
    \end{flushleft}
\end{abstract} 

\selectlanguage{english} 
\begin{abstract}
    Most modern general-purpose programming languages use grammar and syntax that suggests mutable data being an ordinary matter, which in reality complicates reasoning about program state.
    To make such reasoning predictable, this work explores the design and delelopment of a programming language with explicit syntax and semantics of immutable data. 
    This work contains specification for such language and documentation of the architecture for the lexer, parser and AST-walker of the underlying interpreter.
    As a result, a high-level, interpreted, dynamically typed programming language with only atomic data types is created. The language is implemented in C.

    \begin{flushleft}
    \textbf{Keywords:} interpreter, AST, functional programming, C language, data immutability
    \end{flushleft}
\end{abstract} 

\selectlanguage{latvian} 
\tableofcontents

\newpage
\chapter*{Apzīmējumi}
\addcontentsline{toc}{section}{Apzīmējumi} 

\begin{flushleft}

\renewcommand{\arraystretch}{1.3}  % vertical spacing
\begin{tabular}{|l|l|}  % two columns with vertical borders
\hline
\textbf{Saīsinājums} & \textbf{Nozīme} \\  % header row
\hline
AST & Abstraktās sintakses koks\\
\hline
\end{tabular}

\end{flushleft}

\newpage
\chapter*{Ievads}
\addcontentsline{toc}{section}{Ievads} 

Šeit būs ievads

\newpage
\chapter*{Ievads}
\addcontentsline{toc}{section}{Ievads} 
Seit ir ievads

\end{document}

